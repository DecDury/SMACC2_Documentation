This small example shows the steps to follow to trace a SMACC2 library. First, make sure that smacc and tracetools installed (or built in your workspace) in your system.

See README.\+md for setting up a SMACC workspace.

Setting up LTTng


\begin{DoxyCode}{0}
\DoxyCodeLine{sudo apt-\/add-\/repository ppa:lttng/stable-\/2.12}
\DoxyCodeLine{sudo apt-\/get update}
\DoxyCodeLine{sudo apt-\/get install lttng-\/tools lttng-\/modules-\/dkms liblttng-\/ust-\/dev}

\end{DoxyCode}


Create and add your user to a new tracing group


\begin{DoxyCode}{0}
\DoxyCodeLine{newgrp tracing}
\DoxyCodeLine{sudo usermod -\/aG tracing `whoami`}

\end{DoxyCode}
 Reboot or log out and back in to your user to update user groups


\begin{DoxyCode}{0}
\DoxyCodeLine{sudo su \$USER}

\end{DoxyCode}
 Install Python packages for reading and setting up tracing sessions


\begin{DoxyCode}{0}
\DoxyCodeLine{sudo apt-\/get install python3-\/babeltrace python3-\/lttng}

\end{DoxyCode}
 In your SMACC2 workspace, install the following tracing repos.


\begin{DoxyCode}{0}
\DoxyCodeLine{wget https://gitlab.com/ros-\/tracing/ros2\_tracing/raw/master/tracing.repos}
\DoxyCodeLine{vcs import src < tracing.repos}

\end{DoxyCode}
 Finnaly build your workspace


\begin{DoxyCode}{0}
\DoxyCodeLine{colcon build}
\DoxyCodeLine{source install/setup.bash}

\end{DoxyCode}
 Then start your smacc application\+:


\begin{DoxyCode}{0}
\DoxyCodeLine{ros2 launch sm\_three\_some sm\_three\_some.launch}

\end{DoxyCode}
 Once the application is running you can check if the tracepoints are available via the lttng command\+:


\begin{DoxyCode}{0}
\DoxyCodeLine{lttng list -\/-\/userspace}

\end{DoxyCode}
 You should get the list of nodes with tracepoints available. Below it is shown a chunk of the output got with that command, we see how the smacc and the default ros2 tracepoint are available\+:


\begin{DoxyCode}{0}
\DoxyCodeLine{[...]}
\DoxyCodeLine{ID: 1663722 -\/ Name: /home/geus/Desktop/smacc\_tracing/install/sm\_three\_some/lib/sm\_three\_some/sm\_three\_some\_node}
\DoxyCodeLine{      ros2:client\_behavior\_on\_exit\_end (loglevel: TRACE\_DEBUG\_LINE (13)) (type: tracepoint)}
\DoxyCodeLine{      ros2:client\_behavior\_on\_exit\_start (loglevel: TRACE\_DEBUG\_LINE (13)) (type: tracepoint)}
\DoxyCodeLine{      ros2:client\_behavior\_on\_entry\_end (loglevel: TRACE\_DEBUG\_LINE (13)) (type: tracepoint)}
\DoxyCodeLine{      ros2:client\_behavior\_on\_entry\_start (loglevel: TRACE\_DEBUG\_LINE (13)) (type: tracepoint)}
\DoxyCodeLine{      ros2:state\_onExit\_end (loglevel: TRACE\_DEBUG\_LINE (13)) (type: tracepoint)}
\DoxyCodeLine{      ros2:state\_onExit\_start (loglevel: TRACE\_DEBUG\_LINE (13)) (type: tracepoint)}
\DoxyCodeLine{      ros2:state\_onEntry\_end (loglevel: TRACE\_DEBUG\_LINE (13)) (type: tracepoint)}
\DoxyCodeLine{      ros2:state\_onEntry\_start (loglevel: TRACE\_DEBUG\_LINE (13)) (type: tracepoint)}
\DoxyCodeLine{      ros2:state\_onRuntimeConfigure\_end (loglevel: TRACE\_DEBUG\_LINE (13)) (type: tracepoint)}
\DoxyCodeLine{      ros2:state\_onRuntimeConfigure\_start (loglevel: TRACE\_DEBUG\_LINE (13)) (type: tracepoint)}
\DoxyCodeLine{      ros2:update\_end (loglevel: TRACE\_DEBUG\_LINE (13)) (type: tracepoint)}
\DoxyCodeLine{      ros2:update\_start (loglevel: TRACE\_DEBUG\_LINE (13)) (type: tracepoint)}
\DoxyCodeLine{      ros2:smacc\_event (loglevel: TRACE\_DEBUG\_LINE (13)) (type: tracepoint)}
\DoxyCodeLine{      ros2:spinOnce (loglevel: TRACE\_DEBUG\_LINE (13)) (type: tracepoint)}
\DoxyCodeLine{      ros2:callback\_end (loglevel: TRACE\_DEBUG\_LINE (13)) (type: tracepoint)}
\DoxyCodeLine{      ros2:callback\_start (loglevel: TRACE\_DEBUG\_LINE (13)) (type: tracepoint)}
\DoxyCodeLine{      ros2:rclcpp\_callback\_register (loglevel: TRACE\_DEBUG\_LINE (13)) (type: tracepoint)}
\DoxyCodeLine{      ros2:rclcpp\_timer\_callback\_added (loglevel: TRACE\_DEBUG\_LINE (13)) (type: tracepoint)}
\DoxyCodeLine{      ros2:rcl\_timer\_init (loglevel: TRACE\_DEBUG\_LINE (13)) (type: tracepoint)}
\DoxyCodeLine{      ros2:rcl\_client\_init (loglevel: TRACE\_DEBUG\_LINE (13)) (type: tracepoint)}
\DoxyCodeLine{      ros2:rclcpp\_service\_callback\_added (loglevel: TRACE\_DEBUG\_LINE (13)) (type: tracepoint)}
\DoxyCodeLine{      ros2:rcl\_service\_init (loglevel: TRACE\_DEBUG\_LINE (13)) (type: tracepoint)}
\DoxyCodeLine{      ros2:rclcpp\_subscription\_callback\_added (loglevel: TRACE\_DEBUG\_LINE (13)) (type: tracepoint)}
\DoxyCodeLine{      ros2:rclcpp\_subscription\_init (loglevel: TRACE\_DEBUG\_LINE (13)) (type: tracepoint)}
\DoxyCodeLine{      ros2:rcl\_subscription\_init (loglevel: TRACE\_DEBUG\_LINE (13)) (type: tracepoint)}
\DoxyCodeLine{      ros2:rcl\_publisher\_init (loglevel: TRACE\_DEBUG\_LINE (13)) (type: tracepoint)}
\DoxyCodeLine{      ros2:rcl\_node\_init (loglevel: TRACE\_DEBUG\_LINE (13)) (type: tracepoint)}
\DoxyCodeLine{      ros2:rcl\_init (loglevel: TRACE\_DEBUG\_LINE (13)) (type: tracepoint)}
\DoxyCodeLine{      [...]}

\end{DoxyCode}
 Now lets start a recording session. We use a custom command {\ttfamily trace.\+sh}, it is essentially an extended version of {\ttfamily ros2 trace} but also adding the smacc tracepoints to be recorded\+:


\begin{DoxyCode}{0}
\DoxyCodeLine{ros2 run smacc2 trace.sh}

\end{DoxyCode}
 We press a key, record some seconds and then stop the recording pressing again some key\+:


\begin{DoxyCode}{0}
\DoxyCodeLine{ros2 run smacc2 trace.sh}
\DoxyCodeLine{UST tracing enabled (33 events)}
\DoxyCodeLine{kernel tracing enabled (4 events)}
\DoxyCodeLine{context (3 names)}
\DoxyCodeLine{writing tracing session to: /home/geus/.ros/tracing/session-\/20210528232348}
\DoxyCodeLine{press enter to start...}
\DoxyCodeLine{press enter to stop...}
\DoxyCodeLine{stopping \& destroying tracing session}

\end{DoxyCode}
 We can check the tracing database using the babeltrace tool or the tracecompass tool. In our case we use trace compass. Importing the data from the tracing output directory.

 